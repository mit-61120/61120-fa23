From www.sciencemag.org

http://www.sciencemag.org/news/2017/05/descartes-s-brain-had-bulging-frontal-cortex

René Descartes began with doubt. “We cannot 
doubt of our existence while we doubt. … 
I think, therefore I am,” the 17th century 
philosopher and scientist famously wrote. Now, 
modern scientists are trying to figure out what made the 
genius’s mind tick by reconstructing his brain. Scientists 
have long wondered whether the brains of geniuses 
(especially the shapes on their surfaces) could hold 
clues about their owners’ outsized intelligences. But most 
brains studied to date—including Albert Einstein’s—were actual 
brains. Descartes’s had unfortunately decomposed by the time 
scientists wanted to study it. So with techniques normally used 
for studying prehistoric humans, researchers created a 3D image of 
Descartes’s brain (above) by scanning the impression it left 
on the inside of his skull, which has been kept for almost 200 
years now in the National Museum of Natural History in Paris. 
For the most part, his brain was surprisingly normal—its overall 
dimensions fell within regular ranges, compared with 102 other modern 
humans. But one part stood out: an unusual bulge in the frontal cortex, 
in an area which previous studies have suggested may 
process the meaning of words. That’s not to say 
this oddity is necessarily indicative of genius, 
the scientists report online in the Journal of the 
Neurological Sciences. Even Descartes might agree: “It 
is not enough to have a good mind,” he wrote. 
“The main thing is to use it well.”
